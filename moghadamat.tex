\chapter*{مقدمه}
\section{مقدمه}
جبر خطی عددی شاخه‌ای از ریاضیات کاربردی و محاسباتی است که به مطالعه و توسعه الگوریتم‌های عددی برای حل مسائل جبر خطی می‌پردازد. این درس در بسیاری از حوزه‌های علمی و مهندسی نقش کلیدی دارد.

\section{اهداف جبر خطی عددی}


\begin{itemize}
	\item توسعه الگوریتم‌های کارا برای حل دستگاه‌های معادلات خطی.
	\item تقریب و محاسبه مقادیر و بردارهای ویژه ماتریس‌ها.
	\item بررسی پایداری و دقت روش‌های عددی در جبر خطی.
	\item ارائه روش‌هایی برای تجزیه و تحلیل ماتریس‌ها مانند تجزیه LU، تجزیه QR و تجزیه SVD.
	\item کاربرد روش‌های عددی در حل مسائل مهندسی و علمی.
\end{itemize}

\section{کاربردهای جبر خطی عددی}
\begin{itemize}
	\item \textbf{مهندسی:} تحلیل سازه‌ها، پردازش سیگنال و شبیه‌سازی‌های مهندسی.
	\item \textbf{علوم داده و یادگیری ماشین:} کاهش ابعاد داده، الگوریتم‌های بهینه‌سازی و تحلیل داده‌های بزرگ.
	\item \textbf{گرافیک کامپیوتری:} پردازش تصاویر، رندرینگ سه‌بعدی و فشرده‌سازی داده‌ها.
	\item \textbf{اقتصاد و مالی:} مدل‌سازی مالی، بهینه‌سازی سبد سرمایه‌گذاری و تحلیل داده‌های اقتصادی.
	\item \textbf{فیزیک و شیمی محاسباتی:} شبیه‌سازی دینامیک مولکولی و تحلیل سیستم‌های پیچیده.
\end{itemize}
