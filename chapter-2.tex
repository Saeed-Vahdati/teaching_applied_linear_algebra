\chapter{دستگاه معادلات خطی}
\begin{definition}[فرم کلی دستگاه معادلات خطی]
	دستگاه‌های معادلات خطی مجموعه‌ای از معادلات خطی هستند که به دنبال یافتن مقادیر متغیرهایی هستیم که هم‌زمان همه‌ی معادلات را برآورده کنند. فرم کلی یک دستگاه معادلات خطی و نمایش ماتریسی آن به شرح زیر است:
	یک دستگاه معادلات خطی با \( m \) معادله و \( n \) متغیر به صورت زیر نوشته می‌شود:
	
	\[
	\begin{cases}
		a_{11}x_1 + a_{12}x_2 + \dots + a_{1n}x_n = b_1 \\
		a_{21}x_1 + a_{22}x_2 + \dots + a_{2n}x_n = b_2 \\
		\vdots \\
		a_{m1}x_1 + a_{m2}x_2 + \dots + a_{mn}x_n = b_m
	\end{cases}
	\]
	
	که در آن:
	\begin{itemize}
		\item 
		\( x_1, x_2, \dots, x_n \) متغیرهای مجهول هستند.
		\item 
		\( a_{ij} \) ضرایب معلوم هستند (\( i = 1, 2, \dots, m \) و \( j = 1, 2, \dots, n \)).
		\item 
		\( b_1, b_2, \dots, b_m \) مقادیر معلوم (سمت راست معادلات) هستند.
	\end{itemize}

\end{definition}	

\begin{definition}[نمایش ماتریسی دستگاه معادلات خطی]
	دستگاه معادلات خطی فوق را می‌توان به صورت ماتریسی زیر نمایش داد:
	
	\[
	A \mathbf{x} = \mathbf{b}
	\]
	
	که در آن:
	\begin{itemize}
		\item 
		\( A \) ماتریس ضرایب  است و به صورت زیر تعریف می‌شود:
		\[
		A = \begin{pmatrix}
			a_{11} & a_{12} & \dots & a_{1n} \\
			a_{21} & a_{22} & \dots & a_{2n} \\
			\vdots & \vdots & \ddots & \vdots \\
			a_{m1} & a_{m2} & \dots & a_{mn}
		\end{pmatrix}
		\]
		این ماتریس ابعاد \( m \times n \) دارد.
		
		\item 
		\( \mathbf{x} \) بردار متغیرهای مجهول است و به صورت زیر تعریف می‌شود:
		\[
		\mathbf{x} = \begin{pmatrix}
			x_1 \\
			x_2 \\
			\vdots \\
			x_n
		\end{pmatrix}
		\]
		این بردار ابعاد \( n \times 1 \) دارد.
		
		\item 
		\( \mathbf{b} \) بردار مقادیر معلوم (سمت راست معادلات) است و به صورت زیر تعریف می‌شود:
		\[
		\mathbf{b} = \begin{pmatrix}
			b_1 \\
			b_2 \\
			\vdots \\
			b_m
		\end{pmatrix}
		\]
		این بردار ابعاد \( m \times 1 \) دارد.
	\end{itemize}
 
\end{definition}	
\begin{example}
دستگاه معادلات خطی زیر را در نظر بگیرید:
\[
\begin{cases}
	2x_1 + 3x_2 = 5 \\
	4x_1 - x_2 = 1
\end{cases}
\]

این دستگاه را می‌توان به صورت ماتریسی زیر نمایش داد:
\[
\begin{pmatrix}
	2 & 3 \\
	4 & -1
\end{pmatrix}
\begin{pmatrix}
	x_1 \\
	x_2
\end{pmatrix}
=
\begin{pmatrix}
	5 \\
	1
\end{pmatrix}
\]
که در آن:

\begin{itemize}
	\item 
	ماتریس ضرایب \( A \) به صورت زیر است:
	\[
	A = \begin{pmatrix}
		2 & 3 \\
		4 & -1
	\end{pmatrix}
	\]
	
	\item 
	بردار متغیرهای مجهول \( \mathbf{x} \) به صورت زیر است:
	\[
	\mathbf{x} = \begin{pmatrix}
		x_1 \\
		x_2
	\end{pmatrix}
	\]
	
	\item 
	بردار مقادیر معلوم \( \mathbf{b} \) به صورت زیر است:
	\[
	\mathbf{b} = \begin{pmatrix}
		5 \\
		1
	\end{pmatrix}
	\]	
\end{itemize}
\end{example}
\begin{definition}[فرم ماتریس افزوده]
فرم ماتریس افزوده 
برای دستگاه معادلات خطی \( A\mathbf{x} = \mathbf{b} \)، ماتریس افزوده به صورت زیر تعریف می‌شود:
\[
[A \mid \mathbf{b}] = 
\left(
\begin{array}{cccc|c}
	a_{11} & a_{12} & \dots & a_{1n} & b_1 \\
	a_{21} & a_{22} & \dots & a_{2n} & b_2 \\
	\vdots & \vdots & \ddots & \vdots & \vdots \\
	a_{m1} & a_{m2} & \dots & a_{mn} & b_m
\end{array}
\right)
\]
	
\end{definition}
\begin{example}[]
	برای دستگاه:
	\[
	\begin{cases}
		2x + 3y = 5 \\
		4x - y = 1
	\end{cases}
	\]
	ماتریس افزوده:
	\[
	\left(
	\begin{array}{cc|c}
		2 & 3 & 5 \\
		4 & -1 & 1
	\end{array}
	\right)
	\]
\end{example}
 
\begin{definition}[انواع دستگاه‌های معادلات خطی]
	
۱. دستگاه مربعی 

- تعریف: تعداد معادلات \( m \) برابر با تعداد متغیرها \( n \) است (\( m = n \)).

- ویژگی:

- ماتریس ضرایب \( A \) مربعی است.

- اگر \( \det(A) \neq 0 \)، جواب منحصربه‌فرد دارد.

- مثال:
\[
\begin{cases}
	x + 2y = 3 \\
	3x + 4y = 7
\end{cases}
\]

۲. دستگاه فرومعین
 
- تعریف: تعداد معادلات \( m \) کمتر از متغیرها \( n \) است (\( m < n \)).

- ویژگی:

- معمولاً بی‌نهایت جواب دارد.

- ماتریس ضرایب پهن است.

- مثال:
\[
\begin{cases}
	x + y + z = 1 \\
	2x + y - z = 3
\end{cases}
\]

۳. دستگاه فرامعین 

- تعریف: تعداد معادلات \( m \) بیشتر از متغیرها \( n \) است (\( m > n \)).

- ویژگی:

- معمولاً جواب دقیق ندارد (مگر در موارد خاص).

- ماتریس ضرایب بلند است.

- مثال:
\[
\begin{cases}
	x + y = 2 \\
	2x + y = 3 \\
	3x + y = 4
\end{cases}
\]	
\end{definition}
\section{حل دستگاه معادلات خطی}
\subsection{روش ماتریس معکوس}

 حل دستگاه معادلات خطی با استفاده از ماتریس وارون

شرایط استفاده از این روش:\\
- دستگاه باید مربعی باشد (تعداد معادلات = تعداد متغیرها).\\
- ماتریس ضرایب \( A \) باید معکوس‌پذیر باشد (\(\det(A) \neq 0\)).\\


 مراحل حل:\\
1. نمایش ماتریسی دستگاه:\\
\[
A\mathbf{x} = \mathbf{b}
\]
- \( A \): ماتریس ضرایب (\( n \times n \))\\
- \( \mathbf{x} \): بردار مجهولات (\( n \times 1 \))\\
- \( \mathbf{b} \): بردار مقادیر سمت راست (\( n \times 1 \))\\

2. محاسبه ماتریس وارون \( A^{-1} \):\\
- با استفاده از روش‌هایی مانند ماتریس الحاقی یا عملیات سطری.
\\
3. ضرب طرفین در \( A^{-1} \):\\
\[
\mathbf{x} = A^{-1}\mathbf{b}
\]

\begin{example}[]
	دستگاه معادلات:
	\[
	\begin{cases}
		2x + 3y = 5 \\
		4x - y = 1
	\end{cases}
	\]
	
	1. نمایش ماتریسی:
	\[
	\begin{pmatrix}
		2 & 3 \\
		4 & -1
	\end{pmatrix}
	\begin{pmatrix}
		x \\
		y
	\end{pmatrix}
	=
	\begin{pmatrix}
		5 \\
		1
	\end{pmatrix}
	\]
	
	2. محاسبه \( A^{-1} \):
	- دترمینان \( A \):
	\[
	\det(A) = (2)(-1) - (3)(4) = -2 - 12 = -14 \ (\neq 0)
	\]
	- ماتریس الحاقی:
	\[
	\text{adj}(A) = \begin{pmatrix}
		-1 & -3 \\
		-4 & 2
	\end{pmatrix}
	\]
	- ماتریس وارون:
	\[
	A^{-1} = \frac{1}{-14} \begin{pmatrix}
		-1 & -3 \\
		-4 & 2
	\end{pmatrix}
	= \begin{pmatrix}
		\frac{1}{14} & \frac{3}{14} \\
		\frac{2}{7} & -\frac{1}{7}
	\end{pmatrix}
	\]
	
	3. حل دستگاه:
	\[
	\begin{pmatrix}
		x \\
		y
	\end{pmatrix}
	=
	\begin{pmatrix}
		\frac{1}{14} & \frac{3}{14} \\
		\frac{2}{7} & -\frac{1}{7}
	\end{pmatrix}
	\begin{pmatrix}
		5 \\
		1
	\end{pmatrix}
	=
	\begin{pmatrix}
		\frac{5}{14} + \frac{3}{14} \\
		\frac{10}{7} - \frac{1}{7}
	\end{pmatrix}
	=
	\begin{pmatrix}
		\frac{8}{14} \\
		\frac{9}{7}
	\end{pmatrix}
	=
	\begin{pmatrix}
		\frac{4}{7} \\
		\frac{9}{7}
	\end{pmatrix}
	\]
	
	جواب نهایی:
	\[
	x = \frac{4}{7}, \ y = \frac{9}{7}
	\]
\end{example}


 \begin{exercise}
 	سوال۱:
 	
 دستگاه زیر را با روش ماتریس وارون حل کنید:
 \[
 \begin{cases}
 	3x - y = 4 \\
 	2x + 5y = 1
 \end{cases}
 \]
 
 سوال ۲:
 
 آیا دستگاه زیر با این روش قابل حل است؟ چرا؟
 \[
 \begin{cases}
 	x + 2y = 3 \\
 	2x + 4y = 6
 \end{cases}
 \]
 
 سوال ۳:
 
 ماتریس وارون \( A \) را برای دستگاه زیر محاسبه و جواب را بیابید:
 \[
 \begin{cases}
 	1x + 0y + 1z = 2 \\
 	0x + 2y + 0z = 4 \\
 	1x + 1y + 0z = 3
 \end{cases}
 \]	
 \end{exercise}
 \begin{code}[حل دستگاه معادلات خطی به روش ماتریس معکوس]
 	\begin{latin}
 		\lstinputlisting[caption={}]{python_codes/MatrixInverseMethod.py}  
 	\end{latin}
 \end{code}
 \subsection{روش حذفی گاوس}
  حل دستگاه معادلات خطی با روش حذف گاوس 
 
 هدف: تبدیل ماتریس افزوده به فرم سطری پلکانی  یا کاهش‌یافته  برای یافتن جواب.
 

 
\textbf{  مراحل روش حذف گاوس:}\\
\begin{itemize}
\item 
  	 1. تشکیل ماتریس افزوده \([A \mid \mathbf{b}]\).
  	
\item 
  	
  	2. استفاده از عملیات سطری مقدماتی:\\
  	- جابجایی دو سطر.\\
  	- ضرب یک سطر در عددی ناصفر.\\
  	- جمع مضربی از یک سطر با سطر دیگر.
\item 
  	 3. تبدیل به فرم سطری پلکانی.
\item 
  	  4. حل از پایین به بالا.
 
\end{itemize}

\begin{example}[]
	دستگاه مربعی با جواب منحصربه‌فرد
	دستگاه:
	\[
	\begin{cases}
		x + 2y = 5 \\
		3x + 4y = 6
	\end{cases}
	\]
	
	حل:
	1. ماتریس افزوده:
	\[
	\left(
	\begin{array}{cc|c}
		1 & 2 & 5 \\
		3 & 4 & 6
	\end{array}
	\right)
	\]
	2. عملیات سطری:
	- \( R_2 \leftarrow R_2 - 3R_1 \):
	\[
	\left(
	\begin{array}{cc|c}
		1 & 2 & 5 \\
		0 & -2 & -9
	\end{array}
	\right)
	\]
	3. حل:\\
	- از سطر دوم: \(-2y = -9 \Rightarrow y = \frac{9}{2}\).\\
	- از سطر اول: \(x + 2(\frac{9}{2}) = 5 \Rightarrow x = -4\).\\
	
	جواب: \( x = -4 \), \( y = \frac{9}{2} \).
\end{example}
\begin{example}[]
دستگاه فرومعین با بی‌نهایت جواب\\
دستگاه:
\[
\begin{cases}
	x + y + z = 3 \\
	2x + y - z = 1
\end{cases}
\]

حل:\\
1. ماتریس افزوده:
\[
\left(
\begin{array}{ccc|c}
	1 & 1 & 1 & 3 \\
	2 & 1 & -1 & 1
\end{array}
\right)
\]
2. عملیات سطری:
- \( R_2 \leftarrow R_2 - 2R_1 \):
\[
\left(
\begin{array}{ccc|c}
	1 & 1 & 1 & 3 \\
	0 & -1 & -3 & -5
\end{array}
\right)
\]
3. حل:\\
- از سطر دوم: \(-y - 3z = -5 \Rightarrow y = 5 - 3z\).\\
- از سطر اول: \(x + (5 - 3z) + z = 3 \Rightarrow x = -2 + 2z\).\\

جواب عمومی:
\[
x = -2 + 2z, \ y = 5 - 3z \quad (z \in \mathbb{R}).
\]	
\end{example}
\begin{example}[]
دستگاه فرامعین بدون جواب\\
دستگاه:
\[
\begin{cases}
	x + y = 2 \\
	2x + 2y = 5 \\
	3x + 3y = 4
\end{cases}
\]

حل:\\
1. ماتریس افزوده:
\[
\left(
\begin{array}{cc|c}
	1 & 1 & 2 \\
	2 & 2 & 5 \\
	3 & 3 & 4
\end{array}
\right)
\]
2. عملیات سطری:
- \( R_2 \leftarrow R_2 - 2R_1 \):
\[
\left(
\begin{array}{cc|c}
	1 & 1 & 2 \\
	0 & 0 & 1 \\
	3 & 3 & 4
\end{array}
\right)
\]
- سطر دوم نشان‌دهنده‌ی \( 0 = 1 \) است.\\

نتیجه: دستگاه ناسازگار است (جواب ندارد).	
\end{example}
{\scriptsize  \begin{example}[]
	 دستگاه معادلات خطی به صورت:
	
	\[
	\begin{cases}
		x_1 + 3x_2 + x_3 + 5x_4 + x_5 = 5 \\
		x_2 + x_3 + 2x_4 + x_5 = 4 \\
		2x_1 + 4x_2 + 7x_4 + x_5 = 3
	\end{cases}
	\]
	
	 حل با روش حذف گاوس:
	
	مرحله ۱: تشکیل ماتریس افزوده  \\
	ماتریس افزوده برای این دستگاه به صورت زیر است:
	
	\[
	\left[
	\begin{array}{ccccc|c}
		1 & 3 & 1 & 5 & 1 & 5 \\
		0 & 1 & 1 & 2 & 1 & 4 \\
		2 & 4 & 0 & 7 & 1 & 3
	\end{array}
	\right]
	\]
	
	
	مرحله ۲: حذف متغیر \(x_1\) از سطر سوم  \\
	برای حذف \(x_1\) از سطر سوم، از سطر اول استفاده می‌کنیم. عملیات سطری:
	\[
	R_3 \leftarrow R_3 - 2R_1
	\]
	
	ماتریس جدید:
	\[
	\left[
	\begin{array}{ccccc|c}
		1 & 3 & 1 & 5 & 1 & 5 \\
		0 & 1 & 1 & 2 & 1 & 4 \\
		0 & -2 & -2 & -3 & -1 & -7
	\end{array}
	\right]
	\]
	
	
	مرحله ۳: حذف متغیر \(x_2\) از سطر سوم \\ 
	برای حذف \(x_2\) از سطر سوم، از سطر دوم استفاده می‌کنیم. عملیات سطری:
	\[
	R_3 \leftarrow R_3 + 2R_2
	\]
	
	ماتریس جدید:
	\[
	\left[
	\begin{array}{ccccc|c}
		1 & 3 & 1 & 5 & 1 & 5 \\
		0 & 1 & 1 & 2 & 1 & 4 \\
		0 & 0 & 0 & 1 & 1 & 1
	\end{array}
	\right]
	\]
	
	
	مرحله ۴: تحلیل دستگاه  \\
	دستگاه به فرم سطری پلکانی  تبدیل شده است. مشاهده می‌کنیم که:\\
	- متغیرهای اصلی (پایه‌ای): \(x_1\), \(x_2\), \(x_4\) (ستون‌های دارای اولین عدد غیرصفر در هر سطر).\\
	- متغیرهای آزاد: \(x_3\) و \(x_5\) (ستون‌های بدون عدد غیرصفر اصلی).\\
	

	
	مرحله ۵: حل دستگاه به صورت پارامتری  \\
	از سطر سوم شروع می‌کنیم و به صورت پس‌گشت حل می‌کنیم:\\
	
	۱. سطر سوم:
	\[
	x_4 + x_5 = 1 \implies x_4 = 1 - x_5
	\]
	
	۲. سطر دوم:
	\[
	x_2 + x_3 + 2x_4 + x_5 = 4
	\]
	با جایگذاری \(x_4\):
	\[
	x_2 + x_3 + 2(1 - x_5) + x_5 = 4 \implies x_2 + x_3 + 2 - x_5 = 4 \implies x_2 + x_3 - x_5 = 2
	\]
	بنابراین:
	\[
	x_2 = 2 - x_3 + x_5
	\]
	
	۳. سطر اول:
	\[
	x_1 + 3x_2 + x_3 + 5x_4 + x_5 = 5
	\]
	با جایگذاری \(x_2\) و \(x_4\):
	\[
	x_1 + 3(2 - x_3 + x_5) + x_3 + 5(1 - x_5) + x_5 = 5
	\]
	ساده‌سازی:
	\[
	x_1 + 6 - 3x_3 + 3x_5 + x_3 + 5 - 5x_5 + x_5 = 5 \implies x_1 - 2x_3 - x_5 + 11 = 5
	\]
	نتیجه:
	\[
	x_1 = -6 + 2x_3 + x_5
	\]
	
	
	 جواب عمومی دستگاه:
	با توجه به متغیرهای آزاد \(x_3\) و \(x_5\)، جواب به صورت زیر است:
	\[
	\begin{cases}
		x_1 = -6 + 2s + t \\
		x_2 = 2 - s + t \\
		x_3 = s \quad \text{(متغیر آزاد)} \\
		x_4 = 1 - t \\
		x_5 = t \quad \text{(متغیر آزاد)}
	\end{cases}
	\]
	که در آن \(s, t \in \mathbb{R}\) پارامترهای دلخواه هستند.
\end{example}}


\begin{exercise}
 تمرین ۱:
حل کنید:
\[
\begin{cases}
	2x - y = 3 \\
	x + 3y = 5
\end{cases}
\]

تمرین ۲:
حل کنید (در صورت وجود جواب):
\[
\begin{cases}
	x + 2y - z = 1 \\
	2x + 4y - 2z = 2 \\
	3x + 6y - 3z = 3
\end{cases}
\]

تمرین ۳:
حل کنید:
\[
\begin{cases}
	x + y + z = 6 \\
	2y + 5z = -4 \\
	2x + 5y - z = 27
\end{cases}
\]

تمرین ۴:
آیا دستگاه زیر جواب دارد؟
\[
\begin{cases}
	x - y + 2z = 1 \\
	2x + y - z = 0 \\
	3x + 2z = 1
\end{cases}
\]	
\end{exercise} 
\begin{exercise}
	دستگاه معادلات خطی زیر را به روش حذفی گاوس حل کنید.
	
	\[
	\begin{cases}
		2x_1 + 4x_2 - 2x_3 - 2x_4 = -4 \\
		x_1 + 2x_2 + 4x_3 - 3x_4 = 5 \\
		-3x_1 - 3x_2 + 8x_3 - 2x_4 = 7 \\
		-x_1 + x_2 + 6x_3 - 3x_4 = 7
	\end{cases}
	\]
	
\end{exercise}

\begin{code}[حل دستگاه معادلات خطی به روش حذفی گاوس]
	\begin{latin}
		\lstinputlisting[caption={}]{python_codes/GaussianElimination.py}  
	\end{latin}
\end{code}

\begin{exercise}
	یک دستگاه معادلات خطی شامل چهار معادله و چهار مجهول بسازید که بردار جواب آن به صورت 
	$$x=[1,-1,0,2]$$
	باشد. سپس این دستگاه را با استفاده از برنامه‌های پایتون داده شده در قبل حل کنید.
\end{exercise}
\section{روش حذفی گاوس جردن}

روش حذف گاوس-جردن یک الگوریتم برای حل دستگاه‌های معادلات خطی است که ماتریس را به فرم کاهش‌یافته سطری پلکانی  تبدیل می‌کند. این روش، توسعه‌یافته‌ی روش حذف گاوس است و ماتریس را تا حد امکان ساده می‌کند.



 مراحل روش حذف گاوس-جردن:
 
 \begin{itemize}
 	\item [1. ]
 	 تشکیل ماتریس افزوده \([A \mid \mathbf{b}]\).
 	\item [2. ]
 	 تبدیل به فرم سطری پلکانی (REF) با استفاده از عملیات سطری مقدماتی:
 	\begin{itemize}
 		\item [1. ]
 		- جابجایی دو سطر.
 		\item [2. ]
 		- ضرب یک سطر در عددی ناصفر.
 		\item [3. ]
 		- جمع مضربی از یک سطر با سطر دیگر.
 		
 	\end{itemize}
 	\item [3. ]
 	 تبدیل به فرم کاهش‌یافته (RREF):
 	\begin{itemize}
 		\item [1. ]
 	ایجاد ۱‌های اصلی (پایه‌ای) در هر سطر.
 		\item [2. ]
صفر کردن تمام درایه‌های بالا و پایین هر ۱ اصلی.
 		
 	\end{itemize}
 	\item [4. ]
 استخراج جواب از ماتریس کاهش‌یافته.
 	
 \end{itemize}


\begin{example}[]
	دستگاه مربعی با جواب منحصربه‌فرد
	دستگاه:
	\[
	\begin{cases}
		x + 2y = 5 \\
		3x + 4y = 6
	\end{cases}
	\]
	
	حل:
	
	
	1. ماتریس افزوده:
	\[
	\left[
	\begin{array}{cc|c}
		1 & 2 & 5 \\
		3 & 4 & 6
	\end{array}
	\right]
	\]
	2. عملیات سطری:
	- \( R_2 \leftarrow R_2 - 3R_1 \):
	\[
	\left[
	\begin{array}{cc|c}
		1 & 2 & 5 \\
		0 & -2 & -9
	\end{array}
	\right]
	\]
	- \( R_2 \leftarrow -\frac{1}{2}R_2 \):
	\[
	\left[
	\begin{array}{cc|c}
		1 & 2 & 5 \\
		0 & 1 & 4.5
	\end{array}
	\right]
	\]
	- \( R_1 \leftarrow R_1 - 2R_2 \):
	\[
	\left[
	\begin{array}{cc|c}
		1 & 0 & -4 \\
		0 & 1 & 4.5
	\end{array}
	\right]
	\]
	3. جواب:
	\[
	x = -4, \quad y = 4.5
	\]
	
\end{example}

\begin{example}[]
	دستگاه با بی‌نهایت جواب
	دستگاه:
	\[
	\begin{cases}
		x + y + z = 3 \\
		2x + y - z = 1
	\end{cases}
	\]
	
	حل:
	1. ماتریس افزوده:
	\[
	\left[
	\begin{array}{ccc|c}
		1 & 1 & 1 & 3 \\
		2 & 1 & -1 & 1
	\end{array}
	\right]
	\]
	2. عملیات سطری:
	- \( R_2 \leftarrow R_2 - 2R_1 \):
	\[
	\left[
	\begin{array}{ccc|c}
		1 & 1 & 1 & 3 \\
		0 & -1 & -3 & -5
	\end{array}
	\right]
	\]
	- \( R_2 \leftarrow -R_2 \):
	\[
	\left[
	\begin{array}{ccc|c}
		1 & 1 & 1 & 3 \\
		0 & 1 & 3 & 5
	\end{array}
	\right]
	\]
	- \( R_1 \leftarrow R_1 - R_2 \):
	\[
	\left[
	\begin{array}{ccc|c}
		1 & 0 & -2 & -2 \\
		0 & 1 & 3 & 5
	\end{array}
	\right]
	\]
	3. جواب عمومی:
	\[
	x = -2 + 2z, \quad y = 5 - 3z \quad (z \in \mathbb{R})
	\]
	
\end{example}
\begin{exercise}

\begin{itemize}
	\item [1. ]
	دستگاه معادلات زیر را  به روش گاوس-جردن حل کنید(در صورت وجود جواب):
	\[
	\begin{cases}
		2x - y = 3 \\
		x + 3y = 5
	\end{cases}
	\]
	\item [2. ]
		دستگاه معادلات زیر را  به روش گاوس-جردن حل کنید(در صورت وجود جواب):
	\[
	\begin{cases}
		x + 2y - z = 1 \\
		2x + 4y - 2z = 2 \\
		3x + 6y - 3z = 3
	\end{cases}
	\]
		
	\item [3. ]
		دستگاه معادلات زیر را  به روش گاوس-جردن حل کنید(در صورت وجود جواب):
		
	\[
	\begin{cases}
		x + y + z = 6 \\
		2y + 5z = -4 \\
		2x + 5y - z = 27
	\end{cases}
	\]
	
\end{itemize}	
\end{exercise} 
\begin{nokteh}
تفاوت اصلی بین روش حذف گاوس 
\LTRfootnote{Gaussian Elimination}
و روش حذف گاوس-جردن
\LTRfootnote{ Gauss-Jordan Elimination}
در میزان ساده‌سازی ماتریس و نحوه استخراج جواب است. 	
\end{nokteh}	

\begin{nokteh}[هدف نهایی در روش حذفی گاوس]
	\begin{itemize}
		\item 
		ماتریس را به فرم سطری پلکانی
		\LTRfootnote{Row Echelon Form - REF}
		تبدیل می‌کند.  
		\item 
		در REF
		، زیر هر عدد اصلی (پایه‌ای) صفر قرار می‌گیرد.  
		\item 
		جواب با حل پس‌گشت
		\LTRfootnote{Back Substitution} 
		به دست می‌آید.
		
	\end{itemize}
\end{nokteh}
\begin{nokteh}[هدف نهایی در روش حذفی گاوس - جردن]
	


 \begin{itemize}
 	\item 
 	ماتریس را به فرم کاهش‌یافته سطری پلکانی
 	 \LTRfootnote{Reduced Row Echelon Form - RREF }
 	 تبدیل می‌کند.
 	\item 
 	در RREF، هر عدد اصلی ۱ است و تنها عدد غیرصفر در ستون خود است.
 	\item 
 	جواب مستقیماً از ماتریس خوانده می‌شود.
 	
 \end{itemize}
 
\end{nokteh}

\begin{nokteh}[مراحل اجرای روش حذفی گاوس]
	
\begin{itemize}
	\item [1. ]
	ماتریس را به فرم REF می‌آورد.
	\item [2. ]
	با جایگزینی از سطر آخر به بالا، جواب را محاسبه می‌کند.

\end{itemize}	
\end{nokteh}
\begin{nokteh}[مراحل اجرای روش حذفی گاوس-جردن]
	
	\begin{itemize}
		\item [1. ]
		ماتریس را به فرم RREF می‌آورد. 
		\item [2. ]
		جواب بدون نیاز به محاسبات اضافه، مستقیماً از ماتریس استخراج می‌شود.
		
	\end{itemize}	
\end{nokteh}

\begin{nokteh}[نمادگذاری  ماتریس در REF]
	\[
	\begin{bmatrix}
		1 & 2 & 3 & 5 \\
		0 & 1 & 4 & 7 \\
		0 & 0 & 1 & 2
	\end{bmatrix}
	\]
	نیاز به حل معادله‌ی \( z = 2 \)، سپس جایگزینی در معادلات بالاتر.
\end{nokteh}


\begin{nokteh}[نمادگذاری  ماتریس در RREF]
\[
\begin{bmatrix}
	1 & 0 & 0 & -1 \\
	0 & 1 & 0 & 3 \\
	0 & 0 & 1 & 2
\end{bmatrix}
\]
 جواب مستقیماً: \( x = -1 \), \( y = 3 \), \( z = 2 \).
\end{nokteh}

\begin{example}[مثال مقایسه‌ای]
دستگاه معادلات:
\[
\begin{cases}
	2x + y - z = 8 \\
	-3x - y + 2z = -11 \\
	-2x + y + 2z = -3
\end{cases}
\]

حل با حذف گاوس (REF):
ماتریس نهایی:
\[
\begin{bmatrix}
	2 & 1 & -1 & 8 \\
	0 & 0.5 & 0.5 & 1 \\
	0 & 0 & -1 & 1
\end{bmatrix}
\]
- نیاز به حل پس‌گشت برای یافتن \( z \), سپس \( y \), و در نهایت \( x \).

حل با حذف گاوس-جردن (RREF):
ماتریس نهایی:
\[
\begin{bmatrix}
	1 & 0 & 0 & 2 \\
	0 & 1 & 0 & 3 \\
	0 & 0 & 1 & -1
\end{bmatrix}
\]
- جواب مستقیماً: \( x = 2 \), \( y = 3 \), \( z = -1 \).
\end{example}
\begin{exercise}
	تعداد عملیات‌های حسابی (جمع، تفریق، ضرب و تقسیم) را برای هر یک از روش‌های حذفی گاوس و حذفی گاوس جردن محاسبه کنید.
\end{exercise}