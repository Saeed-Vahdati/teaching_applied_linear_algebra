\chapter{دستگاه معادلات خطی}
\begin{definition}[فرم کلی دستگاه معادلات خطی]
	دستگاه‌های معادلات خطی مجموعه‌ای از معادلات خطی هستند که به دنبال یافتن مقادیر متغیرهایی هستیم که هم‌زمان همه‌ی معادلات را برآورده کنند. فرم کلی یک دستگاه معادلات خطی و نمایش ماتریسی آن به شرح زیر است:
	یک دستگاه معادلات خطی با \( m \) معادله و \( n \) متغیر به صورت زیر نوشته می‌شود:
	
	\[
	\begin{cases}
		a_{11}x_1 + a_{12}x_2 + \dots + a_{1n}x_n = b_1 \\
		a_{21}x_1 + a_{22}x_2 + \dots + a_{2n}x_n = b_2 \\
		\vdots \\
		a_{m1}x_1 + a_{m2}x_2 + \dots + a_{mn}x_n = b_m
	\end{cases}
	\]
	
	که در آن:
	\begin{itemize}
		\item 
		\( x_1, x_2, \dots, x_n \) متغیرهای مجهول هستند.
		\item 
		\( a_{ij} \) ضرایب معلوم هستند (\( i = 1, 2, \dots, m \) و \( j = 1, 2, \dots, n \)).
		\item 
		\( b_1, b_2, \dots, b_m \) مقادیر معلوم (سمت راست معادلات) هستند.
	\end{itemize}

\end{definition}	

\begin{definition}[نمایش ماتریسی دستگاه معادلات خطی]
	دستگاه معادلات خطی فوق را می‌توان به صورت ماتریسی زیر نمایش داد:
	
	\[
	A \mathbf{x} = \mathbf{b}
	\]
	
	که در آن:
	\begin{itemize}
		\item 
		\( A \) ماتریس ضرایب  است و به صورت زیر تعریف می‌شود:
		\[
		A = \begin{pmatrix}
			a_{11} & a_{12} & \dots & a_{1n} \\
			a_{21} & a_{22} & \dots & a_{2n} \\
			\vdots & \vdots & \ddots & \vdots \\
			a_{m1} & a_{m2} & \dots & a_{mn}
		\end{pmatrix}
		\]
		این ماتریس ابعاد \( m \times n \) دارد.
		
		\item 
		\( \mathbf{x} \) بردار متغیرهای مجهول است و به صورت زیر تعریف می‌شود:
		\[
		\mathbf{x} = \begin{pmatrix}
			x_1 \\
			x_2 \\
			\vdots \\
			x_n
		\end{pmatrix}
		\]
		این بردار ابعاد \( n \times 1 \) دارد.
		
		\item 
		\( \mathbf{b} \) بردار مقادیر معلوم (سمت راست معادلات) است و به صورت زیر تعریف می‌شود:
		\[
		\mathbf{b} = \begin{pmatrix}
			b_1 \\
			b_2 \\
			\vdots \\
			b_m
		\end{pmatrix}
		\]
		این بردار ابعاد \( m \times 1 \) دارد.
	\end{itemize}
 
\end{definition}	
\begin{example}
دستگاه معادلات خطی زیر را در نظر بگیرید:
\[
\begin{cases}
	2x_1 + 3x_2 = 5 \\
	4x_1 - x_2 = 1
\end{cases}
\]

این دستگاه را می‌توان به صورت ماتریسی زیر نمایش داد:
\[
\begin{pmatrix}
	2 & 3 \\
	4 & -1
\end{pmatrix}
\begin{pmatrix}
	x_1 \\
	x_2
\end{pmatrix}
=
\begin{pmatrix}
	5 \\
	1
\end{pmatrix}
\]
که در آن:

\begin{itemize}
	\item 
	ماتریس ضرایب \( A \) به صورت زیر است:
	\[
	A = \begin{pmatrix}
		2 & 3 \\
		4 & -1
	\end{pmatrix}
	\]
	
	\item 
	بردار متغیرهای مجهول \( \mathbf{x} \) به صورت زیر است:
	\[
	\mathbf{x} = \begin{pmatrix}
		x_1 \\
		x_2
	\end{pmatrix}
	\]
	
	\item 
	بردار مقادیر معلوم \( \mathbf{b} \) به صورت زیر است:
	\[
	\mathbf{b} = \begin{pmatrix}
		5 \\
		1
	\end{pmatrix}
	\]	
\end{itemize}
\end{example}


### فرم ماتریس افزوده (Augmented Matrix)
برای دستگاه معادلات خطی \( A\mathbf{x} = \mathbf{b} \)، ماتریس افزوده به صورت زیر تعریف می‌شود:
\[
[A \mid \mathbf{b}] = 
\left(
\begin{array}{cccc|c}
	a_{11} & a_{12} & \dots & a_{1n} & b_1 \\
	a_{21} & a_{22} & \dots & a_{2n} & b_2 \\
	\vdots & \vdots & \ddots & \vdots & \vdots \\
	a_{m1} & a_{m2} & \dots & a_{mn} & b_m
\end{array}
\right)
\]

**مثال**:
برای دستگاه:
\[
\begin{cases}
	2x + 3y = 5 \\
	4x - y = 1
\end{cases}
\]
ماتریس افزوده:
\[
\left(
\begin{array}{cc|c}
	2 & 3 & 5 \\
	4 & -1 & 1
\end{array}
\right)
\]

---

### انواع دستگاه‌های معادلات خطی

#### ۱. **دستگاه مربعی (Square System)**
- **تعریف**: تعداد معادلات \( m \) برابر با تعداد متغیرها \( n \) است (\( m = n \)).
- **ویژگی**:
- ماتریس ضرایب \( A \) مربعی است.
- اگر \( \det(A) \neq 0 \)، جواب منحصربه‌فرد دارد.
- **مثال**:
\[
\begin{cases}
	x + 2y = 3 \\
	3x + 4y = 7
\end{cases}
\]

#### ۲. **دستگاه فرومعین (Underdetermined System)**
- **تعریف**: تعداد معادلات \( m \) کمتر از متغیرها \( n \) است (\( m < n \)).
- **ویژگی**:
- معمولاً بی‌نهایت جواب دارد.
- ماتریس ضرایب **پهن** است.
- **مثال**:
\[
\begin{cases}
	x + y + z = 1 \\
	2x + y - z = 3
\end{cases}
\]

#### ۳. **دستگاه فرامعین (Overdetermined System)**
- **تعریف**: تعداد معادلات \( m \) بیشتر از متغیرها \( n \) است (\( m > n \)).
- **ویژگی**:
- معمولاً جواب دقیق ندارد (مگر در موارد خاص).
- ماتریس ضرایب **بلند** است.
- **مثال**:
\[
\begin{cases}
	x + y = 2 \\
	2x + y = 3 \\
	3x + y = 4
\end{cases}
\]

---

### شرایط جواب‌دهی دستگاه‌ها
| **نوع دستگاه** | **رنک ماتریس \( A \)** | **رنک ماتریس افزوده \([A \mid \mathbf{b}]\)** | **تعداد جواب‌ها** |
|----------------|------------------------|-----------------------------------------------|-------------------|
| **مربعی**      | \( \text{rank}(A) = n \) | \( \text{rank}([A \mid \mathbf{b}]) = n \)    | ۱ جواب منحصربه‌فرد |
| **فرومعین**    | \( \text{rank}(A) < n \) | \( \text{rank}([A \mid \mathbf{b}]) = \text{rank}(A) \) | بی‌نهایت جواب |
| **فرامعین**    | \( \text{rank}(A) \leq n \) | \( \text{rank}([A \mid \mathbf{b}]) > \text{rank}(A) \) | هیچ جوابی ندارد |

---

### مثال‌های کاربردی
#### دستگاه مربعی با جواب منحصربه‌فرد:
\[
\begin{cases}
	x + y = 2 \\
	x - y = 0
\end{cases}
\]
- جواب: \( x = 1 \), \( y = 1 \).

#### دستگاه فرومعین با بی‌نهایت جواب:
\[
\begin{cases}
	x + y + z = 3 \\
	2x + y - z = 1
\end{cases}
\]
- جواب عمومی: \( x = 2 - 2z \), \( y = 1 + 3z \) (به ازای هر \( z \in \mathbb{R} \)).

#### دستگاه فرامعین بدون جواب:
\[
\begin{cases}
	x + y = 1 \\
	2x + 2y = 3 \\
	3x + 3y = 5
\end{cases}
\]
- ناسازگار است (سطری از ماتریس افزوده به شکل \( [0 \ 0 \mid 1] \) است).

---

### نکات کلیدی
- **رنک ماتریس** (Rank): تعداد سطرهای مستقل خطی در ماتریس.
- **حل‌پذیری**: دستگاه وقتی حل‌پذیر است که \( \text{rank}(A) = \text{rank}([A \mid \mathbf{b}]) \).
- **حل عددی**: برای دستگاه‌های فرامعین از **روش حداقل مربعات** (Least Squares) استفاده می‌شود.

اگر نیاز به توضیح بیشتری دارید، خوشحال می‌شوم کمک کنم! 😊	