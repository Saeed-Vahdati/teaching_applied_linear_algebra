\chapter{بردارها و ماتریس‌ها}
%\begin{landscape} 
\section{تعاریف و مفاهیم پایه‌ای}
\subsection{بردارها}
\begin{definition}[بردار]
یک بردار در فضای $\mathbb{R}^n$ به صورت زیر تعریف می‌شود:
\[ \mathbf{v} = \begin{bmatrix} v_1 \\ v_2 \\ \vdots \\ v_n \end{bmatrix} \]

\begin{example}
مثال: بردار $\mathbf{v} = \begin{bmatrix} 3 \\ -1 \\ 4 \end{bmatrix}$ در $\mathbb{R}^3$.
\end{example}
\end{definition}

\begin{definition}[جمع بردارها]
اگر $\mathbf{u}, \mathbf{v} \in \mathbb{R}^n$ باشند، جمع آن‌ها به‌صورت زیر تعریف می‌شود:
\[ \mathbf{u} + \mathbf{v} = \begin{bmatrix} u_1 + v_1 \\ u_2 + v_2 \\ \vdots \\ u_n + v_n \end{bmatrix} \]
\end{definition}
\begin{definition}[ضرب اسکالر در بردار]
برای یک بردار $\mathbf{v}$ و اسکالر $c$ داریم:
\[ c \mathbf{v} = \begin{bmatrix} c v_1 \\ c v_2 \\ \vdots \\ c v_n \end{bmatrix} \]
\end{definition}
\begin{definition}[ترکیب خطی بردارها]
یک ترکیب خطی از بردارهای $\mathbf{v}_1, \mathbf{v}_2, \dots, \mathbf{v}_k$ به‌صورت زیر تعریف می‌شود:
\[ c_1 \mathbf{v}_1 + c_2 \mathbf{v}_2 + \dots + c_k \mathbf{v}_k \]
\begin{example}
	فرض کنید دو بردار زیر را داشته باشیم:
	\[
	\mathbf{v}_1 = \begin{bmatrix} 1 \\ 2 \\ 3 \end{bmatrix}, \quad
	\mathbf{v}_2 = \begin{bmatrix} 4 \\ 5 \\ 6 \end{bmatrix}
	\]
	اگر ضرایب $c_1 = 2$ و $c_2 = -1$ باشند، ترکیب خطی آن‌ها برابر است با:
	\[
	c_1 \mathbf{v}_1 + c_2 \mathbf{v}_2 =
	2 \begin{bmatrix} 1 \\ 2 \\ 3 \end{bmatrix} +
	(-1) \begin{bmatrix} 4 \\ 5 \\ 6 \end{bmatrix} =
	\begin{bmatrix} 2 \\ 4 \\ 6 \end{bmatrix} +
	\begin{bmatrix} -4 \\ -5 \\ -6 \end{bmatrix} =
	\begin{bmatrix} -2 \\ -1 \\ 0 \end{bmatrix}
	\]
	
\end{example}
\end{definition}
\begin{example}[نوشتن یک بردار به صورت ترکیب خطی چند بردار دیگر]
برداری که به صورت ترکیب خطی دو بردار دیگر نوشته می‌شود:
	
	فرض کنیم سه بردار زیر را داشته باشیم:
	\[
	\mathbf{v}_1 = \begin{bmatrix} 1 \\ 2 \end{bmatrix}, \quad
	\mathbf{v}_2 = \begin{bmatrix} 3 \\ 4 \end{bmatrix}, \quad
	\mathbf{w} = \begin{bmatrix} 7 \\ 10 \end{bmatrix}
	\]
	
	می‌خواهیم ضرایبی مانند \( c_1, c_2 \) پیدا کنیم که:
	\[
	c_1 \mathbf{v}_1 + c_2 \mathbf{v}_2 = \mathbf{w}
	\]
	
	با جایگذاری مقدار بردارها، دستگاه معادلات خطی زیر را داریم:
	\[
	c_1 (1) + c_2 (3) = 7
	\]
	\[
	c_1 (2) + c_2 (4) = 10
	\]
	
	با حل این دستگاه، مقدار \( c_1 = 1 \) و \( c_2 = 2 \) به‌دست می‌آید، پس بردار \( \mathbf{w} \) را می‌توان به صورت ترکیب خطی دو بردار دیگر نوشت.
	
\end{example}

\begin{example}[برداری که به صورت ترکیب خطی دو بردار دیگر نوشته نمی‌شود]

فرض کنیم سه بردار زیر را داشته باشیم:
\[
\mathbf{v}_1 = \begin{bmatrix} 1 \\ 1 \end{bmatrix}, \quad
\mathbf{v}_2 = \begin{bmatrix} 2 \\ 2 \end{bmatrix}, \quad
\mathbf{w} = \begin{bmatrix} 3 \\ 4 \end{bmatrix}
\]

اگر بخواهیم ضرایبی مانند \( c_1, c_2 \) پیدا کنیم که:
\[
c_1 \mathbf{v}_1 + c_2 \mathbf{v}_2 = \mathbf{w}
\]

با جایگذاری مقدار بردارها، دستگاه معادلات خطی زیر را داریم:
\[
c_1 (1) + c_2 (2) = 3
\]
\[
c_1 (1) + c_2 (2) = 4
\]

این دو معادله با هم در تناقض‌اند (چون سمت چپ دو معادله برابر است اما سمت راست متفاوت)، بنابراین چنین ضرایبی وجود ندارد و بردار \( \mathbf{w} \) را نمی‌توان به صورت ترکیب خطی دو بردار دیگر نوشت.
\end{example}

\begin{definition}[ضرب داخلی بردارها]
اگر $\mathbf{u}, \mathbf{v} \in \mathbb{R}^n$ باشند، ضرب داخلی آن‌ها به صورت زیر است:
\[ \mathbf{u} \cdot \mathbf{v} = \sum_{i=1}^{n} u_i v_i \]
\begin{example}
	\[ \mathbf{u} = \begin{bmatrix} 1 \\ 2 \\ 3 \end{bmatrix}, \quad \mathbf{v} = \begin{bmatrix} 4 \\ -5 \\ 6 \end{bmatrix} \Rightarrow \mathbf{u} \cdot \mathbf{v} = (1)(4) + (2)(-5) + (3)(6) = 4 - 10 + 18 = 12 \]
\end{example}
\end{definition}
\begin{definition}[تعمیم ضرب داخلی به بردارهای مختلط]
برای دو بردار مختلط $\mathbf{u}, \mathbf{v} \in \mathbb{C}^n$، ضرب داخلی به صورت زیر تعریف می‌شود:
\[
\mathbf{u} \cdot \mathbf{v} = \sum_{i=1}^{n} u_i \overline{v_i}
\]
که در آن $\overline{v_i}$ مزدوج مختلط $v_i$ است.
\begin{example}
	فرض کنیم دو بردار مختلط زیر را داشته باشیم:
	\[
	\mathbf{u} = \begin{bmatrix} 1 + i \\ 2 - i \end{bmatrix}, \quad
	\mathbf{v} = \begin{bmatrix} 3 - i \\ 1 + 2i \end{bmatrix}
	\]
	ضرب داخلی آن‌ها برابر است با:
	\[
	\mathbf{u} \cdot \mathbf{v} = (1 + i) \overline{(3 - i)} + (2 - i) \overline{(1 + 2i)}
	\]
	\[
	= (1 + i)(3 + i) + (2 - i)(1 - 2i)
	\]
	\[
	= (1 \cdot 3 + 1 \cdot i + i \cdot 3 + i \cdot i) + (2 \cdot 1 + 2 \cdot (-2i) - i \cdot 1 - i \cdot (-2i))
	\]
	\[
	= (3 + i + 3i - 1) + (2 - 4i - i + 2)
	\]
	\[
	= (2 + 4i) + (4 - 5i) = 6 - i
	\]
\end{example}
\end{definition}
\begin{nokteh}

ضرب داخلی بردارها دارای ویژگی‌های مهم زیر است:

1. \textbf{خطی بودن در اولین مؤلفه:}  
برای هر بردارهای $\mathbf{u}, \mathbf{v}, \mathbf{w} \in \mathbb{C}^n$ و ضرایب مختلط $\alpha, \beta$ داریم:
\[
(\alpha \mathbf{u} + \beta \mathbf{v}) \cdot \mathbf{w} = \alpha (\mathbf{u} \cdot \mathbf{w}) + \beta (\mathbf{v} \cdot \mathbf{w})
\]

2. \textbf{خاصیت مزدوج‌گیری:}  
برای هر دو بردار $\mathbf{u}, \mathbf{v} \in \mathbb{C}^n$ داریم:
\[
\mathbf{u} \cdot \mathbf{v} = \overline{\mathbf{v} \cdot \mathbf{u}}
\]
یعنی اگر جای دو بردار را عوض کنیم، مزدوج مختلط نتیجه تغییر می‌کند.

3. \textbf{مثبت معین بودن:}  
برای هر بردار $\mathbf{u} \in \mathbb{C}^n$ داریم:
\[
\mathbf{u} \cdot \mathbf{u} \geq 0
\]
و برابری زمانی رخ می‌دهد که $\mathbf{u} = 0$ باشد.

4. \textbf{نرم بردار:}  
از ضرب داخلی می‌توان برای تعریف \textbf{نرم} (یا طول) یک بردار استفاده کرد:
\[
\|\mathbf{u}\| = \sqrt{\mathbf{u} \cdot \mathbf{u}}
\]

5. \textbf{نامساوی شوارتز:}  
برای هر دو بردار $\mathbf{u}, \mathbf{v} \in \mathbb{C}^n$ داریم:
\[
|\mathbf{u} \cdot \mathbf{v}| \leq \|\mathbf{u}\| \|\mathbf{v}\|
\]
که تعمیم نامساوی کوشی-شوارتز برای بردارهای مختلط است.


\end{nokteh}
\subsection{نُرم بردار}
نُرم یک بردار، یک تابع است که مقدار عددی غیرمنفی را به هر بردار نسبت می‌دهد و نشان‌دهنده اندازه یا طول آن بردار است. به‌طور کلی، نُرم یک بردار $\mathbf{v} \in \mathbb{R}^n$ یا $\mathbb{C}^n$ تابعی است که به‌صورت زیر تعریف می‌شود:

\[
\|\mathbf{v}\|: \mathbb{R}^n \text{ یا } \mathbb{C}^n \to \mathbb{R}^{\geq 0}
\]

و باید سه خاصیت اصلی زیر را داشته باشد:
\begin{nokteh}[خواص نُرم بردار]
	نرم یک بردار باید دارای ویژگی‌های مهم زیر باشد:
	
	1. \textbf{ نامنفی بودن و خاصیت صفر:}  
	برای هر بردار $\mathbf{v} \in \mathbb{C}^n$ داریم:
	\[
	\|\mathbf{v}\| \geq 0, \quad \text{و} \quad \|\mathbf{v}\| = 0 \iff \mathbf{v} = 0
	\]
	
	2. \textbf{همگنی:}  
	برای هر عدد مختلط $\alpha$ و بردار $\mathbf{v} \in \mathbb{C}^n$ داریم:
	\[
	\|\alpha \mathbf{v}\| = |\alpha| \|\mathbf{v}\|
	\]
	یعنی ضرب یک بردار در یک عدد مختلط، مقدار نُرم را به اندازه قدرمطلق آن عدد تغییر می‌دهد.
	
	3. \textbf{نامساوی مثلثی:}  
	برای هر دو بردار $\mathbf{u}, \mathbf{v} \in \mathbb{C}^n$ داریم:
	\[
	\|\mathbf{u} + \mathbf{v}\| \leq \|\mathbf{u}\| + \|\mathbf{v}\|
	\]
	این خاصیت بیان می‌کند که طول مجموع دو بردار از مجموع طول‌های آن‌ها بیشتر نیست.
\end{nokteh}
\begin{definition}[نُرم اقلیدسی]
	نُرم  اقلیدسی یک بردار $\mathbf{v} \in \mathbb{C}^n$ که با $\|\mathbf{v}\|$ نمایش داده می‌شود، به صورت زیر تعریف می‌شود:
	\[
	\|\mathbf{v}\| = \sqrt{\sum_{i=1}^{n} |v_i|^2}
	\]
	که در آن $|v_i|$ مقدار قدرمطلق (یا اندازه) مؤلفه‌های مختلط بردار است.
	به نرم اقلیدسی نُرم ۲ هم گفته می‌شود و بیشتر اوقات آن را با نماد 
	$\|\mathbf{v}\|_2$
	نیز نشان می‌دهند.
\end{definition}

\begin{example}[محاسبه نُرم ۲ یک بردار دو بعدی مختلط مقدار]
فرض کنیم بردار $\mathbf{v} \in \mathbb{C}^2$ به صورت زیر داده شده باشد:
\[
\mathbf{v} = \begin{bmatrix} 3 + 4i \\ 1 - i \end{bmatrix}
\]
نرم این بردار برابر است با:
\[
\|\mathbf{v}\|_2 = \sqrt{|3 + 4i|^2 + |1 - i|^2}
\]
\[
= \sqrt{(3^2 + 4^2) + (1^2 + (-1)^2)}
\]
\[
= \sqrt{(9 + 16) + (1 + 1)}
\]
\[
= \sqrt{27} = 3\sqrt{3}
\]
\end{example}
\begin{definition}[نُرم بی‌نهایت]
	نرم بی‌نهایت یک بردار $\mathbf{v} \in \mathbb{R}^n$ یا $\mathbb{C}^n$ که با $\|\mathbf{v}\|_{\infty}$ نمایش داده می‌شود، به صورت زیر تعریف می‌شود:
	\[
	\|\mathbf{v}\|_{\infty} = \max_{1 \leq i \leq n} |v_i|
	\]
	یعنی بزرگ‌ترین مقدار مطلق در بین مؤلفه‌های بردار را نشان می‌دهد.
\end{definition}
\begin{example}[محاسبه نُرم بی‌نهایت]
	فرض کنیم بردار زیر را داشته باشیم:
	\[
	\mathbf{v} = \begin{bmatrix} -3 \\ 7 \\ 2 \end{bmatrix}
	\]
	در این صورت:
	\[
	\|\mathbf{v}\|_{\infty} = \max \{| -3 |, | 7 |, | 2 | \} = 7
	\]
	
	برای بردار مختلط:
	\[
	\mathbf{w} = \begin{bmatrix} 2 + i \\ -4 - 3i \\ 5 + 2i \end{bmatrix}
	\]
	داریم:
	\[
	\|\mathbf{w}\|_{\infty} = \max \{ |2 + i|, |-4 - 3i|, |5 + 2i| \}
	\]
	\[
	= \max \{ \sqrt{2^2 + 1^2}, \sqrt{(-4)^2 + (-3)^2}, \sqrt{5^2 + 2^2} \}
	\]
	\[
	= \max \{ \sqrt{5}, \sqrt{25}, \sqrt{29} \} = \sqrt{29}
	\]
\end{example}
\begin{definition}[نُرم $p$-ام (\textit{p-Norm})]
نُرم $p$-ام یک بردار $\mathbf{v} \in \mathbb{R}^n$ یا $\mathbb{C}^n$ که با $\|\mathbf{v}\|_p$ نمایش داده می‌شود، به صورت زیر تعریف می‌شود:
\[
\|\mathbf{v}\|_p = \left( \sum_{i=1}^{n} |v_i|^p \right)^{\frac{1}{p}}
\]
که در آن $p \geq 1$ یک عدد حقیقی است.
\end{definition}
\begin{definition}[نُرم $1$ (\textit{ Norm Manhattan})]
نُرم ۱ که به نام نُرم مانهتن یا نُرم تاکسی‌متری نیز شناخته می‌شود، به‌صورت زیر تعریف می‌شود:
\[
\|\mathbf{v}\|_1 = \sum_{i=1}^{n} |v_i|
\]
یعنی مجموع مقادیر قدرمطلق مؤلفه‌های بردار را نشان می‌دهد.

\end{definition}
\begin{example}[محاسبه نرم p]
فرض کنیم بردار $\mathbf{v} \in \mathbb{R}^3$ به صورت زیر داده شده باشد:
\[
\mathbf{v} = \begin{bmatrix} 3 \\ -4 \\ 1 \end{bmatrix}
\]
نُرم 2 (نُرم اقلیدسی) برابر است با:
\[
\|\mathbf{v}\|_2 = \left( |3|^2 + |-4|^2 + |1|^2 \right)^{\frac{1}{2}}
= \left( 9 + 16 + 1 \right)^{\frac{1}{2}} = \sqrt{26}
\]

برای نُرم 3 داریم:
\[
\|\mathbf{v}\|_3 = \left( |3|^3 + |-4|^3 + |1|^3 \right)^{\frac{1}{3}}
= \left( 27 + 64 + 1 \right)^{\frac{1}{3}} = \sqrt[3]{92}
\]

\end{example}
\begin{example}[محاسبه نُرم ۱]
برای بردار:
\[
\mathbf{v} = \begin{bmatrix} 3 \\ -4 \\ 1 \end{bmatrix}
\]
داریم:
\[
\|\mathbf{v}\|_1 = |3| + |-4| + |1| = 3 + 4 + 1 = 8
\]
\end{example}
\begin{example}[محاسبه نُرم ۱ بردار مختلط مقدار]
فرض کنیم بردار $\mathbf{w} \in \mathbb{C}^3$ به صورت زیر داده شده باشد:
\[
\mathbf{w} = \begin{bmatrix} 2 + i \\ -3 - 2i \\ 4i \end{bmatrix}
\]
ابتدا قدرمطلق هر مؤلفه را محاسبه می‌کنیم:
\[
|2 + i| = \sqrt{2^2 + 1^2} = \sqrt{5}
\]
\[
|-3 - 2i| = \sqrt{(-3)^2 + (-2)^2} = \sqrt{9 + 4} = \sqrt{13}
\]
\[
|4i| = \sqrt{(0)^2 + 4^2} = \sqrt{16} = 4
\]

بنابراین، مقدار نُرم ۱ این بردار برابر است با:
\[
\|\mathbf{w}\|_1 = |2 + i| + |-3 - 2i| + |4i|
= \sqrt{5} + \sqrt{13} + 4
\]
\end{example}
\begin{definition}[نرم ضرب داخلی]
از ضرب داخلی می‌توان نُرم یک بردار را محاسبه کرد:
\[
\|\mathbf{v}\| = \sqrt{\mathbf{v} \cdot \mathbf{v}}
\]
\end{definition}

\subsection{ماتریس‌ها}
\begin{definition}[ماتریس]
	یک ماتریس $m \times n$ مجموعه‌ای مستطیلی از اعداد است:
	\[ A = \begin{bmatrix} a_{11} & a_{12} & \dots & a_{1n} \\ a_{21} & a_{22} & \dots & a_{2n} \\ \vdots & \vdots & \ddots & \vdots \\ a_{m1} & a_{m2} & \dots & a_{mn} \end{bmatrix} \]
	
	\begin{example}
		مثال: 
		\[ A = \begin{bmatrix} 1 & 2 \\ 3 & 4 \end{bmatrix} \]
	\end{example}
\end{definition}
\subsection{ماتریس‌های خاص}
	
\begin{definition}[ماتریس مربعی]
	یک ماتریس مربعی، ماتریسی است که تعداد سطرها و ستون‌های آن برابر است:
	\[
	A =
	\begin{bmatrix}
		a_{11} & a_{12} & \dots & a_{1n} \\
		a_{21} & a_{22} & \dots & a_{2n} \\
		\vdots & \vdots & \ddots & \vdots \\
		a_{n1} & a_{n2} & \dots & a_{nn}
	\end{bmatrix}
	\]
	مثال:
	\[
	A =
	\begin{bmatrix}
		2 & -1 & 3 \\
		4 & 0 & -2 \\
		1 & 5 & 7
	\end{bmatrix}
	\]
\end{definition}
\begin{definition}[ماتریس مستطیلی]
	یک ماتریس مستطیلی دارای تعداد سطر و ستون نامساوی است:
	\[
	B =
	\begin{bmatrix}
		b_{11} & b_{12} & b_{13} \\
		b_{21} & b_{22} & b_{23}
	\end{bmatrix}
	\]
	مثال:
	\[
	B =
	\begin{bmatrix}
		1 & 2 & 3 \\
		4 & 5 & 6
	\end{bmatrix}
	\]
\end{definition}
\begin{definition}[ماتریس صفر]
	ماتریسی که تمام درایه‌های آن صفر باشند:
	\[
	O =
	\begin{bmatrix}
		0 & 0 & 0 \\
		0 & 0 & 0 \\
		0 & 0 & 0
	\end{bmatrix}
	\]
\end{definition}
\begin{definition}[ماتریس همانی]
	یک ماتریس مربعی که درایه‌های قطر اصلی آن ۱ و سایر درایه‌ها صفر باشند:
	\[
	I_3=
	\begin{bmatrix}
		1 & 0 & 0 \\
		0 & 1 & 0 \\
		0 & 0 & 1
	\end{bmatrix}
	\]
\end{definition}
\begin{definition}[ماتریس قطری]
	یک ماتریس مربعی که درایه‌های خارج از قطر اصلی آن صفر هستند:
	\[
	D =
	\begin{bmatrix}
		d_1 & 0 & 0 \\
		0 & d_2 & 0 \\
		0 & 0 & d_3
	\end{bmatrix}
	\]
	مثال:
	\[
	D =
	\begin{bmatrix}
		3 & 0 & 0 \\
		0 & -2 & 0 \\
		0 & 0 & 5
	\end{bmatrix}
	\]
\end{definition}
\begin{definition}[ماتریس بالا مثلثی]
	یک ماتریس مربعی که درایه‌های پایین‌تر از قطر اصلی آن صفر هستند:
	\[
	U =
	\begin{bmatrix}
		u_{11} & u_{12} & u_{13} \\
		0 & u_{22} & u_{23} \\
		0 & 0 & u_{33}
	\end{bmatrix}
	\]
	مثال:
	\[
	U =
	\begin{bmatrix}
		2 & -1 & 3 \\
		0 & 4 & 5 \\
		0 & 0 & 6
	\end{bmatrix}
	\]
\end{definition}
\begin{definition}[ماتریس پایین مثلثی]
	یک ماتریس مربعی که درایه‌های بالاتر از قطر اصلی آن صفر هستند:
	\[
	L =
	\begin{bmatrix}
		l_{11} & 0 & 0 \\
		l_{21} & l_{22} & 0 \\
		l_{31} & l_{32} & l_{33}
	\end{bmatrix}
	\]
	مثال:
	\[
	L =
	\begin{bmatrix}
		1 & 0 & 0 \\
		-2 & 3 & 0 \\
		4 & 5 & 6
	\end{bmatrix}
	\]
\end{definition}
\begin{definition}[ماتریس متقارن]
	یک ماتریس مربعی که در آن \( A^T = A \) باشد:
	\[
	S =
	\begin{bmatrix}
		s_{11} & s_{12} & s_{13} \\
		s_{12} & s_{22} & s_{23} \\
		s_{13} & s_{23} & s_{33}
	\end{bmatrix}
	\]
	مثال:
	\[
	S =
	\begin{bmatrix}
		2 & -1 & 3 \\
		-1 & 4 & 5 \\
		3 & 5 & 6
	\end{bmatrix}
	\]
\end{definition}
\begin{definition}[جمع ماتریس‌ها]
	اگر $A, B$ دو ماتریس هم‌اندازه باشند، جمع آن‌ها به صورت زیر تعریف می‌شود:
	\[ (A + B)_{ij} = A_{ij} + B_{ij} \]
	
	
\end{definition}
\begin{definition}[ضرب اسکالر در ماتریس]
	برای یک ماتریس $A$ و اسکالر $c$ داریم:
	\[ (cA)_{ij} = c A_{ij} \]
	
\end{definition}

\begin{definition}[ضرب ماتریس‌ها]
	اگر $A$ یک ماتریس $m \times n$ و $B$ یک ماتریس $n \times p$ باشد، حاصل‌ضرب $AB$ یک ماتریس $m \times p$ است که درایه‌های آن به‌صورت زیر محاسبه می‌شود:
	\[ (AB)_{ij} = \sum_{k=1}^{n} A_{ik} B_{kj} \]
	
\end{definition}
\subsection{نُرم ماتریس}
نُرم یک ماتریس، تعمیمی از نُرم بردار است که اندازه یا بزرگی یک ماتریس را نشان می‌دهد. به‌طور کلی، نُرم ماتریس یک تابع $\|A\|$ است که مقدار عددی غیرمنفی را به هر ماتریس $A$ نسبت می‌دهد و باید خواص زیر را داشته باشد:
\begin{nokteh}[خواص نُرم بردار]
1. نامنفی بودن و خاصیت صفر:
\[
\|A\| \geq 0, \quad \text{و} \quad \|A\| = 0 \iff A = 0
\]

2. همگنی (هم‌ریختی مثبت):
\[
\|\alpha A\| = |\alpha| \|A\|, \quad \forall \alpha \in \mathbb{R} \text{ یا } \mathbb{C}
\]

3. نامساوی مثلثی:
\[
\|A + B\| \leq \|A\| + \|B\|
\]

4. سازگاری با ضرب بردار (در نُرم‌های القایی):
\[
\|A\mathbf{v}\| \leq \|A\| \|\mathbf{v}\|, \quad \forall \mathbf{v} \neq 0
\]
\end{nokteh}
\subsection{انواع نُرم‌های ماتریس}
چندین نُرم برای ماتریس‌ها تعریف می‌شود که بسته به کاربرد مورد استفاده قرار می‌گیرند:
\begin{definition}[ نُرم فروبنیوس (\textit{Norm Frobenius })]
نُرم فروبنیوس، مشابه نُرم اقلیدسی برای بردارها، به‌صورت زیر تعریف می‌شود:
\[
\|A\|_F = \sqrt{\sum_{i=1}^{m} \sum_{j=1}^{n} |a_{ij}|^2}
\]
\textbf{مثال:}  
اگر
\[
A = \begin{bmatrix} 1 & 2 \\ 3 & 4 \end{bmatrix}
\]
باشد، نُرم فروبنیوس برابر است با:
\[
\|A\|_F = \sqrt{1^2 + 2^2 + 3^2 + 4^2} = \sqrt{1 + 4 + 9 + 16} = \sqrt{30}
\]
\end{definition}
\begin{definition}[ نُرم $p$-ام القایی (\textit{Norm Induced })]
نُرم $p$-ام القایی، برای یک ماتریس $A$ به صورت زیر تعریف می‌شود:
\[
\|A\|_p = \sup_{\mathbf{v} \neq 0} \frac{\|A\mathbf{v}\|_p}{\|\mathbf{v}\|_p}
\]

دو حالت خاص آن رایج‌تر هستند:

\textbf{نُرم 1 (نُرم ستون‌محور)}:
\[
\|A\|_1 = \max_{1 \leq j \leq n} \sum_{i=1}^{m} |a_{ij}|
\]
که بیشترین مجموع ستون‌های قدرمطلق را نشان می‌دهد.

\textbf{نُرم $\infty$ (نُرم سطرمحور)}:
\[
\|A\|_{\infty} = \max_{1 \leq i \leq m} \sum_{j=1}^{n} |a_{ij}|
\]
که بیشترین مجموع سطرهای قدرمطلق را نشان می‌دهد.

\textbf{مثال:}  
برای ماتریس:
\[
A = \begin{bmatrix} -2 & 3 \\ 4 & -1 \end{bmatrix}
\]
نُرم 1 برابر است با:
\[
\|A\|_1 = \max\{|-2| + |4|, |3| + |-1|\} = \max\{6, 4\} = 6
\]
و نُرم بی‌نهایت برابر است با:
\[
\|A\|_{\infty} = \max\{|-2| + |3|, |4| + |-1|\} = \max\{5, 5\} = 5
\]
\end{definition}
\begin{definition}[ نُرم طیفی (\textit{  Norm Spectral})]
نُرم طیفی ماتریس $A$ که با $\|A\|_2$ نمایش داده می‌شود، برابر با بزرگترین مقدار ویژه (مقدار تکین) ماتریس است:
\[
\|A\|_2 = \sigma_{\max}(A)
\]
که $\sigma_{\max}(A)$ بزرگ‌ترین مقدار تکین ماتریس $A$ است.

\textbf{مثال:}  
اگر 
\[
A = \begin{bmatrix} 0 & 1 \\ 1 & 0 \end{bmatrix}
\]
باشد، مقادیر ویژه آن $\lambda = \pm 1$ هستند. بنابراین:
\[
\|A\|_2 = \max\{|\lambda_1|, |\lambda_2|\} = \max\{1, 1\} = 1
\]
\end{definition}
\section{ دترمینان}
\begin{definition}[ تعریف دترمینان]
	دترمینان یک ماتریس مربعی را می‌توان به‌صورت بازگشتی با استفاده از \textbf{بسط روی یک سطر یا ستون} محاسبه کرد.  
	دترمینان ماتریس \( A = [a_{ij}] \) مرتبه \( n \) به‌صورت زیر تعریف می‌شود:
	
	\[
	\det(A) = \sum_{j=1}^{n} (-1)^{i+j} a_{ij} M_{ij}
	\]
	
	که در آن
 \( M_{ij} \) دترمینان ماتریس حاصل از حذف سطر \( i \) و ستون \( j \) از \( A \) است.
	
\end{definition}
\begin{nokteh}
	- معمولاً انتخاب **سطر یا ستونی که بیشترین تعداد صفر دارد** محاسبات را ساده‌تر می‌کند.
\end{nokteh}
\begin{example}[دترمینان ماتریس \( 2 \times 2 \)]
	
	\[
	A =
	\begin{bmatrix}
		a & b \\
		c & d
	\end{bmatrix}
	\]
	دترمینان این ماتریس به‌صورت زیر محاسبه می‌شود:
	\[
	\det(A) = ad - bc
	\]
	مثال عددی:
	\[
	A =
	\begin{bmatrix}
		3 & 5 \\
		2 & 7
	\end{bmatrix}
	\]
	\[
	\det(A) = (3)(7) - (5)(2) = 21 - 10 = 11
	\]
\end{example}

	\begin{example}[دترمینان ماتریس \( 3 \times 3 \)]
			\[
		B =
		\begin{bmatrix}
			1 & 2 & 3 \\
			4 & 5 & 6 \\
			7 & 8 & 9
		\end{bmatrix}
		\]
		بسط روی سطر اول:
		\[
		\det(B) = 1 
		\begin{vmatrix}
			5 & 6 \\
			8 & 9
		\end{vmatrix}
		- 2 
		\begin{vmatrix}
			4 & 6 \\
			7 & 9
		\end{vmatrix}
		+ 3 
		\begin{vmatrix}
			4 & 5 \\
			7 & 8
		\end{vmatrix}
		\]
		\[
		= 1(5 \cdot 9 - 6 \cdot 8) - 2(4 \cdot 9 - 6 \cdot 7) + 3(4 \cdot 8 - 5 \cdot 7)
		\]
		\[
		= 1(45 - 48) - 2(36 - 42) + 3(32 - 35)
		\]
		\[
		= 1(-3) - 2(-6) + 3(-3) = -3 + 12 - 9 = 0
		\]
	\end{example}
	پس این ماتریس **دترمینان صفر دارد** و **وابسته خطی** است.
	\begin{example}[دترمینان ماتریس \( 4 \times 4 \)]

	\[
	C =
	\begin{bmatrix}
		2 & 1 & 3 & 4 \\
		1 & 0 & 2 & 1 \\
		3 & 2 & 1 & 0 \\
		4 & 1 & 0 & 2
	\end{bmatrix}
	\]
	بسط روی سطر اول:
	\[
	\det(C) =
	2
	\begin{vmatrix}
		0 & 2 & 1 \\
		2 & 1 & 0 \\
		1 & 0 & 2
	\end{vmatrix}
	- 1
	\begin{vmatrix}
		1 & 2 & 1 \\
		3 & 1 & 0 \\
		4 & 0 & 2
	\end{vmatrix}
	+ 3
	\begin{vmatrix}
		1 & 0 & 1 \\
		3 & 2 & 0 \\
		4 & 1 & 2
	\end{vmatrix}
	- 4
	\begin{vmatrix}
		1 & 0 & 2 \\
		3 & 2 & 1 \\
		4 & 1 & 0
	\end{vmatrix}
	\]
	با محاسبه‌ی دترمینان هر کدام از این ماتریس‌های \(3 \times 3\)، مقدار نهایی به دست می‌آید:
	
	\[
	\det(C) = 2(4) - 1(-3) + 3(5) - 4(2) = 8 + 3 + 15 - 8 = 18
	\]
\end{example}
\subsection{ ماتریس نامنفرد}
\begin{definition}[تعریف ماتریس نامنفرد]
	یک ماتریس مربعی $A \in \mathbb{R}^{n \times n}$ را \textbf{نامنفرد} یا \textbf{معکوس‌پذیر} گویند، اگر و تنها اگر دترمینان آن مخالف صفر باشد، یعنی:
	
	\[
	\det(A) \neq 0
	\]
	
	در این حالت، ماتریس $A$ دارای یک ماتریس معکوس $A^{-1}$ است که  در رابطه زیر صدق می‌کند:
	
	\[
	A A^{-1} = A^{-1} A = I_n
	\]
	که در آن $I_n$ ماتریس همانی مرتبه $n$ است.
\end{definition}
\begin{example}[ماتریس نامنفرد $2 \times 2$]
ماتریس زیر را در نظر بگیرید:
\[
A = \begin{bmatrix} 2 & 3 \\ 1 & 4 \end{bmatrix}
\]
ابتدا دترمینان آن را محاسبه می‌کنیم:
\[
\det(A) = (2 \times 4) - (3 \times 1) = 8 - 3 = 5
\]
چون $\det(A) \neq 0$، این ماتریس نامنفرد است و معکوس‌پذیر می‌باشد.
\end{example}
\begin{example}[ماتریس منفرد $2 \times 2$]
	ماتریس زیر را در نظر بگیرید:
	
	\[
	B = \begin{bmatrix} 2 & 4 \\ 1 & 2 \end{bmatrix}
	\]
	
	دترمینان این ماتریس برابر است با:
	
	\[
	\det(B) = (2 \times 2) - (4 \times 1) = 4 - 4 = 0
	\]
	
	چون $\det(B) = 0$، این ماتریس **منفرد** است و معکوس ندارد.
\end{example}
\begin{example}[ماتریس نامنفرد $3 \times 3$]
	در نظر بگیرید:
	
	\[
	C = \begin{bmatrix} 1 & 2 & 3 \\ 0 & 1 & 4 \\ 5 & 6 & 0 \end{bmatrix}
	\]
	
	دترمینان این ماتریس را به کمک بسط محاسبه می‌کنیم:
	
	\[
	\det(C) = 1 \times \begin{vmatrix} 1 & 4 \\ 6 & 0 \end{vmatrix}
	- 2 \times \begin{vmatrix} 0 & 4 \\ 5 & 0 \end{vmatrix}
	+ 3 \times \begin{vmatrix} 0 & 1 \\ 5 & 6 \end{vmatrix}
	\]
	
	\[
	= 1 \times (1 \times 0 - 4 \times 6)
	- 2 \times (0 \times 0 - 4 \times 5)
	+ 3 \times (0 \times 6 - 1 \times 5)
	\]
	
	\[
	= 1 \times (-24) - 2 \times (-20) + 3 \times (-5) = -24 + 40 - 15 = 1
	\]
	
	چون $\det(C) \neq 0$، ماتریس $C$ نامنفرد و معکوس‌پذیر است.
\end{example}
	
	\section{خواص دترمینان}
	دترمینان یک ماتریس دارای ویژگی‌های مهمی است که در ادامه به برخی از این خواص اشاره می‌کنیم:
	
	\begin{enumerate}
		\item \textbf{دترمینان ماتریس همانی:} 
		\[
		\det(I_n) = 1
		\]
		که در آن $I_n$ ماتریس همانی مرتبه $n$ است.
		
		\item \textbf{دترمینان ماتریس ناصفر فقط برای ماتریس نامنفرد:}
		\[
		\det(A) \neq 0 \quad \Rightarrow \quad A \text{ معکوس‌پذیر است.}
		\]
		
		\item \textbf{دترمینان ماتریس منفرد برابر صفر است:}
		\[
		\det(A) = 0 \quad \Rightarrow \quad A \text{ ماتریس منفرد است و معکوس ندارد.}
		\]
		
		\item \textbf{خاصیت ضرب دترمینان:}
		برای دو ماتریس مربعی هم‌مرتبه $A$ و $B$ داریم:
		\[
		\det(AB) = \det(A) \det(B).
		\]
		
		\item \textbf{دترمینان ماتریس معکوس:}
		اگر $A$ یک ماتریس نامنفرد باشد، آنگاه:
		\[
		\det(A^{-1}) = \frac{1}{\det(A)}.
		\]
		
		\item \textbf{دترمینان ماتریس بالامثلثی یا پایین‌مثلثی:}
		اگر $A$ یک ماتریس مثلثی (بالامثلثی یا پایین‌مثلثی) باشد، دترمینان آن برابر است با حاصل‌ضرب درایه‌های قطری:
		\[
		\det(A) = a_{11} a_{22} \dots a_{nn}.
		\]
		
		\item \textbf{اثر ضرب یک سطر یا ستون در یک عدد ثابت:}
		اگر تمام درایه‌های یک سطر یا ستون ماتریس $A$ در عدد ثابت $c$ ضرب شوند، دترمینان نیز در همان مقدار ضرب می‌شود:
		\[
		\det(B) = c \det(A).
		\]
		
		\item \textbf{جابجایی دو سطر یا دو ستون:}
		اگر در یک ماتریس دو سطر یا دو ستون را با هم جابجا کنیم، دترمینان علامت عوض می‌کند:
		\[
		\det(A') = -\det(A).
		\]
		
		\item \textbf{سطرها یا ستون‌های مساوی یا مضرب یکدیگر:}
		اگر دو سطر یا دو ستون یک ماتریس مساوی باشند یا مضرب یکدیگر باشند، آنگاه:
		\[
		\det(A) = 0.
		\]
		
		\item \textbf{دترمینان ترانهاده‌ی ماتریس:}
		برای هر ماتریس مربعی $A$ داریم:
		\[
		\det(A^T) = \det(A).
		\]
		
	\end{enumerate}
		
			
\section{کدهای پایتون}
\begin{code}[تعریف بردار و ماتریس در پایتون]
	در سطر سوم برنامه زیر یک بردار و در سطر پنجم یک ماتریس در محیط پایتون تعریف شده‌اند. برای تعریف این دو نیاز به کتابخانه numpy وجود دارد که در سطر اول این کتابخانه وارد و از np به عنوان اختصار آن استفاده شده است.
	\begin{latin}
		\lstinputlisting[caption={}]{python_codes/VectorAndMatrix.py}  
	\end{latin}
\end{code}
\begin{code}[محاسبه نُرم‌های مختلف یک ماتریس در پایتون]
	\begin{latin}
		\lstinputlisting[caption={}]{python_codes/MatrixNorm.py} 
		\texttt{Norm 1 (Column Norm): 9.0\\
		Infinity Norm (Row Norm): 8.0\\
		Frobenius Norm: 7.810249675906654\\
		Spectral Norm (Largest Singular Value): 6.40854048954407 }
	\end{latin}
\end{code}

\begin{code}[محاسبه تقریبی نُرم 1 یک ماتریس با استفاده از هزار بردار تصادفی غیر صفر]
	\begin{latin}
		\lstinputlisting[caption={}]{python_codes/DefMatrixNorm1.py}  
		\texttt{Approximate Norm 1: 8.396903386801652}
	\end{latin}
\end{code}
\begin{code}[محاسبه دترمینان یک ماتریس]
	\begin{latin}
		\lstinputlisting[caption={}]{python_codes/DetMatrix.py}  
\texttt{Determinant of matrix A:\\
-286.0}
	\end{latin}
\end{code}
\section{تمرین‌ها}
\begin{exercise}
	\begin{itemize}
		\item 
برای ماتریس‌های صفر و ماتریس‌های همانی انواع نُرم‌های ماتریسی را بدست آورید. چه نتیجه‌ای می‌توان گرفت؟
\item 
ماتریس زیر را در نظر بگیرید هر یک از نُرم‌های ماتریس روی آن را محاسبه کنید.
\[
D =
\begin{bmatrix}
	d_1 & 0 & 0 & \dots & 0 \\
	0 & d_2 & 0 & \dots & 0 \\
	0 & 0 & d_3 & \dots & 0 \\
	\vdots & \vdots & \vdots & \ddots & \vdots \\
	0 & 0 & 0 & \dots & d_n
\end{bmatrix}
\]
\end{itemize}
\end{exercise}
%\end{landscape} 